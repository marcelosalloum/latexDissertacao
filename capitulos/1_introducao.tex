\chapter{Introdu\c{c}\~ao}
Mecanismos \'opticos, tais como o olho humano ou c\^ameras de v\'ideo, funcionam atrav\'es do mapeamento de uma cena tridimensional em um plano bidimensional, seja ele a retina do olho, um filme fotogr\'afico ou um receptor fotossens\'ivel.

Um sistema de vis\~ao computacional percorre o caminho inverso, extraindo informa\c{c}\~oes --- geralmente tridimensionais --- a partir de imagens bidimensionais \cite{Gross87}. Para tal, \'e necess\'ario controlar determinadas vari\'aveis do sistema a fim de isolar um n\'umero menor de dados que ser\~ao utilizados para inferir algumas das dimens\~oes presentes na cena.

Os m\'etodos tradicionais para aquisi\c{c}\~ao da profundidade incluem sonares e lasers, que possuem limita\c{c}\~ao na latitude de exposi\c{c}\~ao, e m\'etodos de processamento de imagens tais como a paralaxe e a vis\~ao est\'ereo, que s\~ao limitados em avaliar pequenas diferen\c{c}as de profundidade devido \`a ambiguidade de correspond\^encias \cite{Darrell88}, uma vez que a informa\c{c}\~ao percebida n\~ao \'e suficiente para que seja obtida uma solu\c{c}\~ao \'unica e exata para qualquer entrada \cite{Oliveira06}.

Entretanto, informa\c{c}\~oes sobre a profundidade dos objetos de uma cena podem ser recuperadas atrav\'es de t\'ecnicas binoculares n\~ao intrusivas, sem que haja problemas de ambiguidade na correspond\^encia. Particularmente, a profundidade pode ser computada a partir do foco, analisando uma cena enquanto varia-se a configura\c{c}\~ao focal e determinando o foco correto para cada objeto da cena. Dada uma c\^amera com foco ajust\'avel existe uma correspond\^encia do tipo um-para-um entre a posi\c{c}\~ao da lente e a dist\^ancia ao plano de foco. Portanto, a configura\c{c}\~ao do foco determina a profundidade.